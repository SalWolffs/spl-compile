\documentclass{beamer}
\setbeamercovered{transparent}
\usepackage{xcolor}


\title{Lexical Analysis}
\author{Sal Wolffs (s4064542)}
\institute{Radboud University Nijmegen}
\date{2021-02-23}

\newcommand\heading[1]{%
  \par\bigskip
  {\large\bfseries#1}\par\smallskip}

\definecolor{green}{RGB}{0,200,0}

\begin{document}
\begin{frame}
\titlepage
\end{frame}

\begin{frame}
\frametitle{Implementation Language: Rust}
    \begin{itemize}[<+(1)->]
        \item Modern systems programming language
        \item Inspired by C++, Haskell, and OCaml, among others
        \item \color{green} \textbf{+} Strong memory safety
        \item \color{red} \hspace{1mm}\textbf{-}\hspace{1mm} Verbose
        \item \color{green} \textbf{+} Cheap\&expressive abstractions
        \item \color{green} \textbf{+} Sound type system
        \item \color{red} \hspace{1mm}\textbf{-}\hspace{1mm} Not as composable as a true functional language
        \item \color{red} \hspace{1mm}\textbf{-}\hspace{1mm} Relatively long clean compilation times 
        \item \color{green} \textbf{+} Fully-featured official toolchain utility \texttt{cargo}
    \end{itemize}
\end{frame}

\begin{frame}
    \frametitle{Grammar Adjustments}
    \begin{itemize}[<+(1)->]
        \item Inline or introduce some rules, left-factor, and regularize
            \begin{itemize}[<+(1)->]
                \item New non-terminals, e.g. \texttt{Compound = `\{'
                    Stmt\textsuperscript{*} `\}' }
                \item Removed e.g. \texttt{[FTypes]}, replaced by
                    \texttt{Type\textsuperscript{*}}
            \end{itemize}
        \item Rewrite Exp to use numerical precedence
        \item Generally simplify by accepting a bit more (e.g. bad
            \texttt{Void})
            \begin{itemize}[<+(1)->]
                \item Semantic analysis can reject nonsense
            \end{itemize}
        \item Create separate "parsing grammar" for LL(1) compatibility
        \item Remove negative integer literals until constant folding
        \item Character literals: lisp-style, no redundant closers
        \item In parsing, unify all tuple-like constructions (\texttt{(Exp),
            (Exp,Exp), (argid,argid,argid)}...)
    \end{itemize}
\end{frame}

\begin{frame}
    \frametitle{Lexer Design}
    \begin{itemize}[<+(1)->]
        \item Communication: Lexer object is an ``Iterator''
            \begin{itemize}[<+(1)->]
                \item Same convention as in Python
                \item Similar to sequentially generated lists in Haskell
            \end{itemize}
        \item Main logic: massive \texttt{match} statement
        \item String interning integrated into the lexer
            \begin{itemize}[<+(1)->]
                \item Lookup array as a public field
                \item Trivializes keyword recognition: preload hashmap
            \end{itemize}
        \item Special case: dedicated recursive balanced \texttt{/* */} matcher
        \item Comments are skipped entirely by lexer
            \begin{itemize}[<+(1)->]
                \item Simplifies parsing
                \item Complicates using pretty printer as beautifier
            \end{itemize}
    \end{itemize}
\end{frame}

\begin{frame}
    \frametitle{AST\&Parser Design}
    \begin{itemize}[<+(1)->]
        \item Manual \textit{predictive} LL(1) recursive descent parser: linear
            time.
        \item Lookahead 1: lexer implements \texttt{peek}
        \item AST structure mimics grammar
        \item Parser call structure mimics parsing grammar
        \item Problem: Expressions aren't LL(1) 
            \begin{itemize}[<+(1)->]
                \item Delegate to shunting yard parser, calls main parser for atoms
                \item Atoms: variable, literal, call, tuple, parenthesized expression
            \end{itemize}
    \end{itemize}
\end{frame}

\begin{frame}
    \frametitle{Error Handling}
    \begin{itemize}[<+(1)->]
        \item Tokens annotated with line, length, and column
        \item AST nodes annotated with start and end line and column
        \item Internal errors (compiler bugs) \textit{panic}, yielding a backtrace
        \item Both lexer and parser return \texttt{Result} types
            \begin{itemize}[<+(1)->]
                \item Syntax errors generate \texttt{Result::Err} values
                \item \texttt{Err} values are propagated up explicitly (think \texttt{Either}
                    monad)
            \end{itemize}
        \item Currently no error recovery, could be added
            \begin{itemize}[<+(1)->]
                \item Hardest problem: recovery points. Maybe `;' and balanced `\}', depending on rule
                \item Could easily log \texttt{Err} to \texttt{Vec} before
                    recovery
            \end{itemize}
    \end{itemize}
\end{frame}

\begin{frame}
    \frametitle{Automated Testing}
    \begin{itemize}[<+(1)->]
        \item Easily accomplished using \texttt{cargo test}
        \item \texttt{\#[test]} annotates unit and integration tests
        \item Uncaught panic (from e.g. failed assert) fails test
        \item \texttt{\#[cfg(test)]} excepts test modules from production build
        \item Git hooks run tests before every commit, warn on broken code
    \end{itemize}
\end{frame}

\begin{frame}
    \frametitle{Reflections and Improvements}
    \begin{itemize}[<+(1)->]
        \item Excepting education, use libraries to skip this stage:
            \begin{itemize}[<+(1)->]
                \item \texttt{logos} for lexer generation
                \item \texttt{lasso} for string interning
                \item \texttt{lalrpop} to generate a fast LR(1) parser
            \end{itemize}
        \item Error accumulation might be nice. Repair sounds too DWIM though.
        \item To make the pretty printer a useful formatter, maybe store
            comments (with location) separately and re-insert them on printing.
        \item It should be relatively easy to support n-tuples fully. 
        \item Currently, errors are a string and a location. Better: enum with a
            pretty printer.
    \end{itemize}
\end{frame}

\begin{frame}[plain,c]
    \begin{center}
    \Huge{Questions}
    \end{center}
\end{frame}
\end{document}

